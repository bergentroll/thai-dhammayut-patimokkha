\section{Nissaggiyapācittiyā}
\label{np}

\begin{intro}
  Ime kho pan'āyasmanto tiṁsa nissaggiyā pācittiyā dhammā uddesaṁ āgacchanti.
\end{intro}

\setsubsecheadstyle{\subsectionFmt}
\subsection{Cīvaravaggo}
\vspace{0.2cm}

\pdfbookmark[3]{Nissaggiya Pācittiya 1}{np1}
\subsubsection*{\hyperref[forf-exp1]{Nissaggiya Pācittiya 1: Kaṭhinasikkhāpadaṁ}}
\label{np1}

Niṭṭhita-cīvarasmiṁ bhikkhunā ubbhatasmiṁ kaṭhine, dasāha-paramaṁ atireka-cīvaraṁ dhāretabbaṁ. Taṁ atikkāmayato, nissaggiyaṁ pācittiyaṁ.

\pdfbookmark[3]{Nissaggiya Pācittiya 2}{np2}
\subsubsection*{\hyperref[forf-exp2]{Nissaggiya Pācittiya 2: Uddositasikkhāpadaṁ}}
\label{np2}

Niṭṭhita-cīvarasmiṁ bhikkhunā ubbhatasmiṁ kaṭhine, eka-rattam'pi ce bhikkhu ti-cīvarena vippavaseyya, aññatra bhikkhu-sammatiyā, nissaggiyaṁ pācittiyaṁ.

\pdfbookmark[3]{Nissaggiya Pācittiya 3}{np3}
\subsubsection*{\hyperref[forf-exp3]{Nissaggiya Pācittiya 3: Akālacīvarasikkhāpadaṁ}}
\label{np3}

Niṭṭhita-cīvarasmiṁ bhikkhunā ubbhatasmiṁ kaṭhine, bhikkhuno pan'eva akāla-cīvaraṁ uppajjeyya, ākaṅkhamānena bhikkhunā paṭiggahetabbaṁ. Paṭiggahetvā khippam'eva kāretabbaṁ. No c'assa pāripūri, māsa-paraman'tena bhikkhunā taṁ cīvaraṁ nikkhipitabbaṁ, ūnassa pāripūriyā satiyā paccāsāya. Tato ce uttariṁ nikkhipeyya satiyā'pi paccāsāya, nissaggiyaṁ pācittiyaṁ.

\pdfbookmark[3]{Nissaggiya Pācittiya 4}{np4}
\subsubsection*{\hyperref[forf-exp4]{Nissaggiya Pācittiya 4: Purāṇacīvarasikkhāpadaṁ}}
\label{np4}

Yo pana bhikkhu aññātikāya bhikkhuniyā purāṇa-cīvaraṁ dhovāpeyya vā rajāpeyya vā ākoṭāpeyya vā, nissaggiyaṁ pācittiyaṁ.

\pdfbookmark[3]{Nissaggiya Pācittiya 5}{np5}
\subsubsection*{\hyperref[forf-exp5]{Nissaggiya Pācittiya 5: Cīvarappaṭiggahaṇasikkhāpadaṁ}}
\label{np5}

Yo pana bhikkhu aññātikāya bhikkhuniyā hatthato cīvaraṁ paṭiggaṇheyya aññatra pārivaṭṭakā, nissaggiyaṁ pācittiyaṁ.

\pdfbookmark[3]{Nissaggiya Pācittiya 6}{np6}
\subsubsection*{\hyperref[forf-exp6]{Nissaggiya Pācittiya 6: Aññātakaviññattisikkhāpadaṁ}}
\label{np6}

Yo pana bhikkhu aññātakaṁ gahapatiṁ vā gahapatāniṁ vā cīvaraṁ viññāpeyya aññatra samayā, nissaggiyaṁ pācittiyaṁ. Tatth'āyaṁ samayo: Acchinna-cīvaro vā hoti bhikkhu naṭṭha-cīvaro vā. Ayaṁ tattha samayo.

\pdfbookmark[3]{Nissaggiya Pācittiya 7}{np7}
\subsubsection*{\hyperref[forf-exp7]{Nissaggiya Pācittiya 7: Tat'uttarisikkhāpadaṁ}}
\label{np7}

Tañ-ce aññātako gahapati vā gahapatānī vā bahūhi cīvarehi abhihaṭṭhum-pavāreyya, santar'uttara-paraman'tena bhikkhunā tato cīvaraṁ sāditabbaṁ. Tato ce uttariṁ sādiyeyya, nissaggiyaṁ pācittiyaṁ.

\pdfbookmark[3]{Nissaggiya Pācittiya 8}{np8}
\subsubsection*{\hyperref[forf-exp8]{Nissaggiya Pācittiya 8: Paṭhama-upakkhaṭasikkhāpadaṁ}}
\label{np8}

Bhikkhuṁ pan'eva uddissa aññātakassa gahapatissa vā gahapatāniyā vā cīvara-cetāpanaṁ upakkhaṭaṁ hoti, “Iminā cīvara-cetāpanena cīvaraṁ cetāpetvā itthan'nāmaṁ bhikkhuṁ cīvarena acchādessāmī” ti. Tatra ce so bhikkhu pubbe appavārito upasaṅkamitvā cīvare vikappaṁ āpajjeyya, “Sādhu vata maṁ āyasmā iminā cīvara-cetāpanena, evarūpaṁ vā evarūpaṁ vā cīvaraṁ cetāpetvā acchādehī” ti, kalyāṇa-kamyataṁ upādāya, nissaggiyaṁ pācittiyaṁ.

\pdfbookmark[3]{Nissaggiya Pācittiya 9}{np9}
\subsubsection*{\hyperref[forf-exp9]{Nissaggiya Pācittiya 9: Dutiya-upakkhaṭasikkhāpadaṁ}}
\label{np9}

ikkhuṁ pan'eva uddissa ubhinnaṁ aññātakānaṁ gahapatīnaṁ vā gahapatānīnaṁ vā pacceka-cīvara-cetāpanā upakkhaṭā honti, “Imehi mayaṁ pacceka-cīvara-cetāpanehi pacceka-cīvarāni cetāpetvā itthan'nāmaṁ bhikkhuṁ cīvarehi acchādessāmā” ti. Tatra ce so bhikkhu pubbe appavārito upasaṅkamitvā cīvare vikappaṁ āpajjeyya, “Sādhu vata maṁ āyasmanto imehi pacceka-cīvara-cetāpanehi, evarūpaṁ vā evarūpaṁ vā cīvaraṁ cetāpetvā acchādetha ubho'va santā ekenā ” ti, kalyāṇa-kamyataṁ upādāya, nissaggiyaṁ pācittiyaṁ.

\pdfbookmark[3]{Nissaggiya Pācittiya 10}{np10}
\subsubsection*{\hyperref[forf-exp10]{Nissaggiya Pācittiya 10: Rājasikkhāpadaṁ}}
\label{np10}

Bhikkhuṁ pan'eva uddissa rājā vā rājabhoggo vā brāhmaṇo vā gahapatiko vā dūtena cīvara-cetāpanaṁ pahiṇeyya, “Iminā cīvara-cetāpanena cīvaraṁ cetāpetvā itthan'nāmaṁ bhikkhuṁ cīvarena acchādehī” ti. So ce dūto taṁ bhikkhuṁ upasaṅkamitvā evaṁ vadeyya, “Idaṁ kho bhante āyasmantaṁ uddissa cīvara-cetāpanaṁ ābhataṁ. Paṭiggaṇhātu āyasmā cīvara-cetāpanan” ti. Tena bhikkhunā so dūto evam-assa vacanīyo, “Na kho mayaṁ āvuso cīvara-cetāpanaṁ paṭiggaṇhāma, cīvarañ-ca kho mayaṁ paṭiggaṇhāma kālena kappiyan” ti. So ce dūto taṁ bhikkhuṁ evaṁ vadeyya, “Atthi pan'āyasmato koci veyyāvaccakaro” ti. Cīvar'atthikena bhikkhave bhikkhunā veyyāvaccakaro niddisitabbo, ārāmiko vā upāsako vā, “Eso kho āvuso bhikkhūnaṁ veyyāvaccakaro” ti. So ce dūto taṁ veyyāvaccakaraṁ saññāpetvā taṁ bhikkhuṁ upasaṅkamitvā evaṁ vadeyya, “Yaṁ kho bhante āyasmā veyyāvaccakaraṁ niddisi, saññatto so mayā. Upasaṅkamatu āyasmā kālena cīvarena taṁ acchādessatī” ti. Cīvar'atthikena bhikkhave bhikkhunā veyyāvaccakaro upasaṅkamitvā dvittikkhattuṁ codetabbo sāretabbo, “Attho me āvuso cīvarenā” ti. Dvittikkhattuṁ codayamāno sārayamāno taṁ cīvaraṁ abhinipphādeyya, icc'etaṁ kusalaṁ. No ce abhinipphādeyya, catukkhattuṁ pañcakkhattuṁ chakkhattu-paramaṁ tuṇhī-bhūtena uddissa ṭhātabbaṁ. Catukkhattuṁ pañcakkhattuṁ chakkhattu-paramaṁ tuṇhī-bhūto uddissa tiṭṭhamāno taṁ cīvaraṁ abhinipphādeyya, icc'etaṁ kusalaṁ. No ce abhinipphādeyya, tato ce uttariṁ vāyamamāno taṁ cīvaraṁ abhinipphādeyya, nissaggiyaṁ pācittiyaṁ. No ce abhinipphādeyya, yatassa cīvara-cetāpanaṁ ābhataṁ, tattha sāmaṁ vā gantabbaṁ, dūto vā pāhetabbo, “Yaṁ kho tumhe āyasmanto bhikkhuṁ uddissa cīvara-cetāpanaṁ pahiṇittha. Na tan-tassa bhikkhuno kiñci atthaṁ anubhoti. Yuñjant'āyasmanto sakaṁ. Mā vo sakaṁ vinassī” ti. Ayaṁ tattha sāmīci.

\begin{center}
  Cīvaravaggo paṭhamo
\end{center}

\subsection{Eḷakalomavaggo}
\vspace{0.2cm}

\pdfbookmark[3]{Nissaggiya Pācittiya 11}{np11}
\subsubsection*{\hyperref[forf-exp11]{Nissaggiya Pācittiya 11: Kosiyasikkhāpadaṁ}}
\label{np11}

Yo pana bhikkhu kosiya-missakaṁ santhataṁ kārāpeyya, nissaggiyaṁ pācittiyaṁ.

\pdfbookmark[3]{Nissaggiya Pācittiya 12}{np12}
\subsubsection*{\hyperref[forf-exp12]{Nissaggiya Pācittiya 12: Suddhakāḷakasikkhāpadaṁ}}
\label{np12}

Yo pana bhikkhu suddha-kāḷakānaṁ eḷaka-lomānaṁ santhataṁ kārāpeyya, nissaggiyaṁ pācittiyaṁ.

\pdfbookmark[3]{Nissaggiya Pācittiya 13}{np13}
\subsubsection*{\hyperref[forf-exp13]{Nissaggiya Pācittiya 13: Dvebhāgasikkhāpadaṁ}}
\label{np13}

Navam-pana bhikkhunā santhataṁ kārayamānena, dve bhāgā suddha-kāḷakānaṁ eḷaka-lomānaṁ ādātabbā, tatiyaṁ odātānaṁ catutthaṁ gocariyānaṁ. Anādā ce bhikkhu dve bhāge suddha-kāḷakānaṁ eḷaka-lomānaṁ, tatiyaṁ odātānaṁ catutthaṁ gocariyānaṁ navaṁ santhataṁ kārāpeyya, nissaggiyaṁ pācittiyaṁ.

\pdfbookmark[3]{Nissaggiya Pācittiya 14}{np14}
\subsubsection*{\hyperref[forf-exp14]{Nissaggiya Pācittiya 14: Chabbassasikkhāpadaṁ}}
\label{np14}

Navam-pana bhikkhunā santhataṁ kārāpetvā chabbassāni dhāretabbaṁ. Orena ce channaṁ vassānaṁ taṁ santhataṁ vissajjetvā vā avissajjetvā vā aññaṁ navaṁ santhataṁ kārāpeyya, aññatra bhikkhu-sammatiyā, nissaggiyaṁ pācittiyaṁ.

\pdfbookmark[3]{Nissaggiya Pācittiya 15}{np15}
\subsubsection*{\hyperref[forf-exp15]{Nissaggiya Pācittiya 15: Nisīdanasanthatasikkhāpadaṁ}}
\label{np15}

Nisīdana-santhatam-pana bhikkhunā kārayamānena purāṇa-santhatassa sāmantā sugata-vidatthi ādātabbā dubbaṇṇa-karaṇāya. Anādā ce bhikkhu purāṇa-santhatassa sāmantā sugata-vidatthiṁ navaṁ nisīdana-santhataṁ kārāpeyya, nissaggiyaṁ pācittiyaṁ.

\pdfbookmark[3]{Nissaggiya Pācittiya 16}{np16}
\subsubsection*{\hyperref[forf-exp16]{Nissaggiya Pācittiya 16: Eḷakalomasikkhāpadaṁ}}
\label{np16}

Bhikkhuno pan'eva addhāna-magga-paṭipannassa eḷaka-lomāni uppajjeyyuṁ. Ākaṅkhamānena bhikkhunā paṭiggahetabbāni. Paṭiggahetvā ti-yojana-paramaṁ sahatthā hāretabbāni, asante hārake. Tato ce uttariṁ hareyya asante'pi hārake, nissaggiyaṁ pācittiyaṁ.

\pdfbookmark[3]{Nissaggiya Pācittiya 17}{np17}
\subsubsection*{\hyperref[forf-exp17]{Nissaggiya Pācittiya 17: Eḷakalomadhovāpanasikkhāpadaṁ}}
\label{np17}

Yo pana bhikkhu aññātikāya bhikkhuniyā eḷaka-lomāni dhovāpeyya vā rajāpeyya vā vijaṭāpeyya vā, nissaggiyaṁ pācittiyaṁ.

\pdfbookmark[3]{Nissaggiya Pācittiya 18}{np18}
\subsubsection*{\hyperref[forf-exp18]{Nissaggiya Pācittiya 18: Rūpiyasikkhāpadaṁ}}
\label{np18}

Yo pana bhikkhu jātarūpa-rajataṁ uggaṇheyya vā uggaṇhāpeyya vā upanikkhittaṁ vā sādiyeyya, nissaggiyaṁ pācittiyaṁ.

\pdfbookmark[3]{Nissaggiya Pācittiya 19}{np19}
\subsubsection*{\hyperref[forf-exp19]{Nissaggiya Pācittiya 19: Rūpiyasaṁvohārasikkhāpadaṁ}}
\label{np19}

Yo pana bhikkhu nānappakārakaṁ rūpiya-saṁvohāraṁ samāpajjeyya, nissaggiyaṁ pācittiyaṁ.

\pdfbookmark[3]{Nissaggiya Pācittiya 20}{np20}
\subsubsection*{\hyperref[forf-exp20]{Nissaggiya Pācittiya 20: Kayavikkayasikkhāpadaṁ}}
\label{np20}

Yo pana bhikkhu nānappakārakaṁ kaya-vikkayaṁ samāpajjeyya, nissaggiyaṁ pācittiyaṁ.

\begin{center}
  Kosiya-vaggo dutiyo.
\end{center}

\subsection{Pattavaggo}
\vspace{0.2cm}

\pdfbookmark[3]{Nissaggiya Pācittiya 21}{np21}
\subsubsection*{\hyperref[forf-exp21]{Nissaggiya Pācittiya 21: Pattasikkhāpadaṁ}}
\label{np21}

Dasāha-paramaṁ atireka-patto dhāretabbo. Taṁ atikkāmayato, nissaggiyaṁ pācittiyaṁ.

\pdfbookmark[3]{Nissaggiya Pācittiya 22}{np22}
\subsubsection*{\hyperref[forf-exp22]{Nissaggiya Pācittiya 22: Ūnapañcabandhanasikkhāpadaṁ}}
\label{np22}

Yo pana bhikkhu ūna-pañca-bandhanena pattena aññaṁ navaṁ pattaṁ cetāpeyya, nissaggiyaṁ pācittiyaṁ. Tena bhikkhunā so patto bhikkhu-parisāya nissajjitabbo. Yo ca tassā bhikkhu-parisāya patta-pariyanto, so ca tassa bhikkhuno padātabbo, “Ayan-te bhikkhu patto, yāva bhedanāya dhāretabbo” ti. Ayaṁ tattha sāmīci.

\pdfbookmark[3]{Nissaggiya Pācittiya 23}{np23}
\subsubsection*{\hyperref[forf-exp23]{Nissaggiya Pācittiya 23: Bhesajjasikkhāpadaṁ}}
\label{np23}

Yāni kho pana tāni gilānānaṁ bhikkhūnaṁ paṭisāyanīyāni bhesajjāni, seyyathīdaṁ: sappi navanītaṁ telaṁ madhu phāṇitaṁ; tāni paṭiggahetvā sattāha-paramaṁ sannidhi-kārakaṁ paribhuñjitabbāni. Taṁ atikkāmayato, nissaggiyaṁ pācittiyaṁ.

\pdfbookmark[3]{Nissaggiya Pācittiya 24}{np24}
\subsubsection*{\hyperref[forf-exp24]{Nissaggiya Pācittiya 24: Vassikasāṭikasikkhāpadaṁ}}
\label{np24}

“Māso seso gimhānan” ti bhikkhunā vassika-sāṭika-cīvaraṁ pariyesitabbaṁ. “Aḍḍha-māso seso gimhānan” ti katvā nivāsetabbaṁ. “Orena ce māso seso gimhānan” ti vassika-sāṭika-cīvaraṁ pariyeseyya, “Oren'aḍḍha-māso seso gimhānan” ti katvā nivāseyya, nissaggiyaṁ pācittiyaṁ.

\pdfbookmark[3]{Nissaggiya Pācittiya 25}{np25}
\subsubsection*{\hyperref[forf-exp25]{Nissaggiya Pācittiya 25: Cīvara-acchindanasikkhāpadaṁ}}
\label{np25}

Yo pana bhikkhu bhikkhussa sāmaṁ cīvaraṁ datvā kupito anattamano acchindeyya vā acchindāpeyya vā, nissaggiyaṁ pācittiyaṁ.

\pdfbookmark[3]{Nissaggiya Pācittiya 26}{np26}
\subsubsection*{\hyperref[forf-exp26]{Nissaggiya Pācittiya 26: Suttaviññattisikkhāpadaṁ}}
\label{np26}

Yo pana bhikkhu sāmaṁ suttaṁ viññāpetvā tantavāyehi cīvaraṁ vāyāpeyya, nissaggiyaṁ pācittiyaṁ.

\pdfbookmark[3]{Nissaggiya Pācittiya 27}{np27}
\subsubsection*{\hyperref[forf-exp27]{Nissaggiya Pācittiya 27: Mahāpesakārasikkhāpadaṁ}}
\label{np27}

Bhikkhuṁ pan'eva uddissa aññātako gahapati vā gahapatānī vā tantavāyehi cīvaraṁ vāyāpeyya. Tatra ce so bhikkhu pubbe appavārito tantavāye upasaṅkamitvā cīvare vikappaṁ āpajjeyya, “Idaṁ kho āvuso cīvaraṁ maṁ uddissa vīyati. Āyatañ-ca karotha vitthatañ-ca appitañ-ca suvītañ-ca supavāyitañ-ca suvilekhitañ-ca suvitacchitañ-ca karotha; app'eva nāma mayam'pi āyasmantānaṁ kiñci-mattaṁ anupadajjeyyāmā” ti. Evañ-ca so bhikkhu vatvā kiñci-mattaṁ anupadajjeyya, antamaso piṇḍapāta-mattam-pi, nissaggiyaṁ pācittiyaṁ.

\pdfbookmark[3]{Nissaggiya Pācittiya 28}{np28}
\subsubsection*{\hyperref[forf-exp28]{Nissaggiya Pācittiya 28: Accekacīvarasikkhāpadaṁ}}
\label{np28}

Das'āh'ānāgataṁ kattika-temāsi-puṇṇamaṁ, bhikkhuno pan'eva acceka-cīvaraṁ uppajjeyya. Accekaṁ maññamānena bhikkhunā paṭiggahetabbaṁ. Paṭiggahetvā yāva cīvara-kāla-samayaṁ nikkhipitabbaṁ. Tato ce uttariṁ nikkhipeyya, nissaggiyaṁ pācittiyaṁ.

\pdfbookmark[3]{Nissaggiya Pācittiya 29}{np29}
\subsubsection*{\hyperref[forf-exp29]{Nissaggiya Pācittiya 29: Sāsaṅkasikkhāpadaṁ}}
\label{np29}

Upavassaṁ kho pana kattika-puṇṇamaṁ. Yāni kho pana tāni āraññakāni sen'āsanāni sāsaṅka-sammatāni sappaṭibhayāni, tathā-rūpesu bhikkhu sen'āsanesu viharanto, ākaṅkhamāno tiṇṇaṁ cīvarānaṁ aññataraṁ cīvaraṁ antara-ghare nikkhipeyya. Siyā ca tassa bhikkhuno kocid'eva paccayo tena cīvarena vippavāsāya, chāratta-paraman-tena bhikkhunā tena cīvarena vippavasitabbaṁ. Tato ce uttariṁ vippavaseyya, aññatra bhikkhu-sammatiyā, nissaggiyaṁ pācittiyaṁ.

\pdfbookmark[3]{Nissaggiya Pācittiya 30}{np30}
\subsubsection*{\hyperref[forf-exp30]{Nissaggiya Pācittiya 30: Pariṇatasikkhāpadaṁ}}
\label{np30}

Yo pana bhikkhu jānaṁ saṅghikaṁ lābhaṁ pariṇataṁ attano pariṇāmeyya, nissaggiyaṁ pācittiyaṁ.

\begin{center}
  Patta-vaggo tatiyo.
\end{center}

\medskip

\begin{center}
Uddiṭṭhā kho āyasmanto tiṁsa nissaggiyā pācittiyā dhammā.

\smallskip

Tatth'āyasmante pucchāmi: Kacci'ttha parisuddhā?\\
Dutiyam'pi pucchāmi: Kacci'ttha parisuddhā?\\
Tatiyam'pi pucchāmi: Kacci'ttha parisuddhā?

\smallskip

Parisuddh'etth'āyasmanto, tasmā tuṇhī, evam'etaṁ dhārayāmi.
\end{center}

\begin{outro}
  Nissaggiyā pācittiyā dhammā niṭṭhitā
\end{outro}

\clearpage
