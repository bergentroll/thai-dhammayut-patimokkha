\section{Pāṭidesanīyā}
\label{pd}

\begin{intro}
  Ime kho pan'āyasmanto cattāro pāṭidesanīyā dhammā uddesaṁ āgacchanti.
\end{intro}

\setsubsecheadstyle{\subsubsectionFmt}
\pdfbookmark[3]{Pāṭidesanīyā 1}{pd1}
\subsection*{\hyperref[ack1]{Pāṭidesanīyā 1: Paṭhamapāṭidesanīyasikkhāpadaṁ}}
\label{pd1}

Yo pana bhikkhu aññātikāya bhikkhuniyā antara-gharaṁ paviṭṭhāya hatthato, khādanīyaṁ vā bhojanīyaṁ vā sahatthā paṭiggahetvā khādeyya vā bhuñjeyya vā, paṭidesetabbaṁ tena bhikkhunā, “Gārayhaṁ āvuso dhammaṁ āpajjiṁ asappāyaṁ pāṭidesanīyaṁ, taṁ paṭidesemī” ti.

\pdfbookmark[3]{Pāṭidesanīyā 2}{pd2}
\subsection*{\hyperref[ack2]{Pāṭidesanīyā 2: Dutiyapāṭidesanīyasikkhāpadaṁ}}
\label{pd2}

Bhikkhū pan'eva kulesu nimantitā bhuñjanti. Tatra ce bhikkhunī vosāsamāna-rūpā ṭhitā hoti, “Idha sūpaṁ detha, idha odanaṁ dethā” ti. Tehi bhikkhūhi sā bhikkhunī apasādetabbā, “Apasakka tāva bhagini, yāva bhikkhū bhuñjantī” ti. Ekassa'pi ce bhikkhuno nappaṭibhāseyya taṁ bhikkhuniṁ apasādetuṁ, “Apasakka tāva bhagini, yāva bhikkhū bhuñjantī” ti, paṭidesetabbaṁ tehi bhikkhūhi, “Gārayhaṁ āvuso dhammaṁ āpajjimhā asappāyaṁ pāṭidesanīyaṁ, taṁ paṭidesemā” ti.

\pdfbookmark[3]{Pāṭidesanīyā 3}{pd3}
\subsection*{\hyperref[ack3]{Pāṭidesanīyā 3: Tatiyapāṭidesanīyasikkhāpadaṁ}}
\label{pd3}

Yāni kho pana tāni sekkha-sammatāni kulāni. Yo pana bhikkhu tathā-rūpesu sekkha-sammatesu kulesu pubbe animantito agilāno khādanīyaṁ vā bhojanīyaṁ vā sahatthā paṭiggahetvā khādeyya vā bhuñjeyya vā, paṭidesetabbaṁ tena bhikkhunā, “Gārayhaṁ āvuso dhammaṁ āpajjiṁ asappāyaṁ pāṭidesanīyaṁ, taṁ paṭidesemī” ti.

\pdfbookmark[3]{Pāṭidesanīyā 4}{pd4}
\subsection*{\hyperref[ack4]{Pāṭidesanīyā 4: Catutthapāṭidesanīyasikkhāpadaṁ}}
\label{pd4}

Yāni kho pana tāni āraññakāni sen'āsanāni sāsaṅka-sammatāni sappaṭibhayāni. Yo pana bhikkhu tathā-rūpesu sen'āsanesu viharanto, pubbe appaṭisaṁviditaṁ khādanīyaṁ vā bhojanīyaṁ vā ajjhārāme sahatthā paṭiggahetvā agilāno khādeyya vā bhuñjeyya vā, paṭidesetabbaṁ tena bhikkhunā, “Gārayhaṁ āvuso dhammaṁ āpajjiṁ asappāyaṁ pāṭidesanīyaṁ, taṁ paṭidesemī” ti.

\medskip

\begin{center}
Uddiṭṭhā kho āyasmanto cattāro pāṭidesanīyā dhammā.

\smallskip

Tatth'āyasmante pucchāmi: Kacci'ttha parisuddhā?\\
Dutiyam'pi pucchāmi: Kacci'ttha parisuddhā?\\
Tatiyam'pi pucchāmi: Kacci'ttha parisuddhā?

\smallskip

Parisuddh'etth'āyasmanto, tasmā tuṇhī, evam'etaṁ dhārayāmi.
\end{center}

\begin{outro}
  Pāṭidesanīyā niṭṭhitā
\end{outro}

\clearpage
