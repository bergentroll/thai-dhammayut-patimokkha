\setsecheadstyle{\sectionFmt}
\section{Saṅghādises'uddeso}
\label{sd}

\begin{intro}
  Ime kho pan'āyasmanto terasa saṅghādisesā dhammā uddesaṁ āgacchanti.
\end{intro}

\pdfbookmark[2]{Saṅghādisesa 1}{sd1}
\subsection*{\hyperref[comm1]{Saṅghādisesa 1: Sukkavissaṭṭhisikkhāpadaṁ}}
\label{sd1}
Sañcetanikā sukka-visaṭṭhi aññatra supinantā, saṅghādiseso.

\pdfbookmark[2]{Saṅghādisesa 2}{sd2}
\subsection*{\hyperref[comm2]{Saṅghādisesa 2: Kāyasaṁsaggasikkhāpadaṁ}}
\label{sd2}
Yo pana bhikkhu otiṇṇo vipariṇatena cittena mātugāmena saddhiṁ kāya-saṁsaggaṁ samāpajjeyya, hattha-gāhaṁ vā veṇi-gāhaṁ vā aññatarassa vā aññatarassa vā aṅgassa parāmasanaṁ, saṅghādiseso.

\pdfbookmark[2]{Saṅghādisesa 3}{sd3}
\subsection*{\hyperref[comm3]{Saṅghādisesa 3: Duṭṭhullavācāsikkhāpadaṁ}}
\label{sd3}
Yo pana bhikkhu otiṇṇo vipariṇatena cittena mātugāmaṁ duṭṭhullāhi vācāhi obhāseyya, yathā taṁ yuvā yuvatiṁ methunūpasañhitāhi, saṅghādiseso.

\pdfbookmark[2]{Saṅghādisesa 4}{sd4}
\subsection*{\hyperref[comm4]{Saṅghādisesa 4: Attakāmapāricariyasikkhāpadaṁ}}
\label{sd4}
Yo pana bhikkhu otiṇṇo vipariṇatena cittena mātugāmassa santike atta-kāma-pāricariyāya vaṇṇaṁ bhāseyya, “Etad-aggaṁ bhagini pāricariyānaṁ, yā m’ādisaṁ sīlavantaṁ kalyāṇa-dhammaṁ brahmacāriṁ etena dhammena paricareyyā” ti, methunūpasañhitena, saṅghādiseso.

\pdfbookmark[2]{Saṅghādisesa 5}{sd5}
\subsection*{\hyperref[comm5]{Saṅghādisesa 5: Sañcarittasikkhāpadaṁ}}
\label{sd5}
Yo pana bhikkhu sañcarittaṁ samāpajjeyya, itthiyā vā purisa-matiṁ, purisassa vā itthī-matiṁ, jāyattane vā jārattane vā antamaso taṁ-khaṇikāya-pi, saṅghādiseso.

\pdfbookmark[2]{Saṅghādisesa 6}{sd6}
\subsection*{\hyperref[comm6]{Saṅghādisesa 6: Kuṭikārasikkhāpadaṁ}}
\label{sd6}
Saññācikāya pana bhikkhunā kuṭiṁ kārayamānena assāmikaṁ att’uddesaṁ pamāṇikā kāretabbā. Tatr’idaṁ pamāṇaṁ: dīghaso dvādasa vidatthiyo sugata-vidatthiyā, tiriyaṁ satt’antarā. Bhikkhū abhinetabbā vatthu-desanāya. Tehi bhikkhūhi vatthuṁ desetabbaṁ anārambhaṁ saparikkamanaṁ. Sārambhe ce bhikkhu vatthusmiṁ aparikkamane saññācikāya kuṭiṁ kāreyya, bhikkhū vā anabhineyya vatthu-desanāya, pamāṇaṁ vā atikkāmeyya, saṅghādiseso.

\pdfbookmark[2]{Saṅghādisesa 7}{sd7}
\subsection*{\hyperref[comm7]{Saṅghādisesa 7: Vihārakārasikkhāpadaṁ}}
\label{sd7}
Mahallakam-pana bhikkhunā vihāraṁ kārayamānena, sassāmikaṁ att’uddesaṁ bhikkhū abhinetabbā vatthu-desanāya. Tehi bhikkhūhi vatthuṁ desetabbaṁ anārambhaṁ saparikkamanaṁ. Sārambhe ce bhikkhu vatthusmiṁ aparikkamane mahallakaṁ vihāraṁ kāreyya, bhikkhū vā anabhineyya vatthu-desanāya, saṅghādiseso.

\pdfbookmark[2]{Saṅghādisesa 8}{sd8}
\subsection*{\hyperref[comm8]{Saṅghādisesa 8: Duṭṭhadosasikkhāpadaṁ}}
\label{sd8}
Yo pana bhikkhu bhikkhuṁ duṭṭho doso appatīto amūlakena pārājikena dhammena anuddhaṁseyya, “App’eva nāma naṁ imamhā brahmacariyā cāveyyan” ti. Tato aparena samayena samanuggāhiyamāno vā asamanuggāhiyamāno vā, amūlakañ-c’eva taṁ adhikaraṇaṁ hoti, bhikkhu ca dosaṁ patiṭṭhāti, saṅghādiseso.

\pdfbookmark[2]{Saṅghādisesa 9}{sd9}
\subsection*{\hyperref[comm9]{Saṅghādisesa 9: Aññabhāgiyasikkhāpadaṁ}}
\label{sd9}
Yo pana bhikkhu bhikkhuṁ duṭṭho doso appatīto aññabhāgiyassa adhikaraṇassa kiñci desaṁ lesa-mattaṁ upādāya pārājikena dhammena anuddhaṁseyya, “App’eva nāma naṁ imamhā brahmacariyā cāveyyan” ti. Tato aparena samayena samanuggāhiyamāno vā asamanuggāhiyamāno vā, aññabhāgiyañ-c’eva taṁ adhikaraṇaṁ hoti, koci deso lesa-matto upādinno, bhikkhu ca dosaṁ patiṭṭhāti, saṅghādiseso.

\pdfbookmark[2]{Saṅghādisesa 10}{sd10}
\subsection*{\hyperref[comm10]{Saṅghādisesa 10: Saṅghabhedasikkhāpadaṁ}}
\label{sd10}
Yo pana bhikkhu samaggassa saṅghassa bhedāya parakkameyya, bhedana-saṁvattanikaṁ vā adhikaraṇaṁ samādāya paggayha tiṭṭheyya, so bhikkhu bhikkhūhi evam-assa vacanīyo, “Mā āyasmā samaggassa saṅghassa bhedāya parakkami. Bhedana-saṁvattanikaṁ vā adhikaraṇaṁ samādāya paggayha aṭṭhāsi. Samet’āyasmā saṅghena, samaggo hi saṅgho sammodamāno avivadamāno ek’uddeso phāsu viharatī” ti. Evañ-ca so bhikkhu bhikkhūhi vuccamāno tath’eva paggaṇheyya, so bhikkhu bhikkhūhi yāva-tatiyaṁ samanubhāsitabbo tassa paṭinissaggāya. Yāva-tatiyañ-ce samanubhāsiyamāno taṁ paṭinissajjeyya, icc’etaṁ kusalaṁ. No ce paṭinissajjeyya, saṅghādiseso.

\pdfbookmark[2]{Saṅghādisesa 11}{sd11}
\subsection*{\hyperref[comm11]{Saṅghādisesa 11: Bhed'ānuvattakasikkhāpadaṁ}}
\label{sd11}
Tass’eva kho pana bhikkhussa bhikkhū honti anuvattakā vagga-vādakā, eko vā dve vā tayo vā, te evaṁ vadeyyuṁ, “Mā āyasmanto etaṁ bhikkhuṁ kiñci avacuttha. Dhamma-vādī c’eso bhikkhu, vinaya-vādī c’eso bhikkhu, amhākañ-c’eso bhikkhu chandañ-ca ruciñ-ca ādāya voharati. Jānāti no bhāsati, amhākam-p’etaṁ khamatī” ti. Te bhikkhū bhikkhūhi evam-assu vacanīyā, “Mā āyasmanto evaṁ avacuttha. Na c’eso bhikkhu dhamma-vādī, na c’eso bhikkhu vinaya-vādī. Mā āyasmantānam'pi saṅgha-bhedo rucittha. Samet‘'āyasmantānaṁ saṅghena, samaggo hi saṅgho sammodamāno avivadamāno ek’uddeso phāsu viharatī” ti. Evañ-ca te bhikkhū bhikkhūhi vuccamānā tath’eva paggaṇheyyuṁ, te bhikkhū bhikkhūhi yāva-tatiyaṁ samanubhāsitabbā tassa paṭinissaggāya. Yāva-tatiyañ-ce samanubhāsiyamānā taṁ paṭinissajjeyyuṁ, icc’etaṁ kusalaṁ. No ce paṭinissajjeyyuṁ, saṅghādiseso.

\pdfbookmark[2]{Saṅghādisesa 12}{sd12}
\subsection*{\hyperref[comm12]{Saṅghādisesa 12: Dubbacasikkhāpadaṁ}}
\label{sd12}
Bhikkhu pan’eva dubbaca-jātiko hoti, uddesa-pariyāpannesu sikkhāpadesu bhikkhūhi saha-dhammikaṁ vuccamāno attānaṁ avacanīyaṁ karoti, “Mā maṁ āyasmanto kiñci avacuttha kalyāṇaṁ vā pāpakaṁ vā. Aham-p’āyasmante na kiñci vakkhāmi kalyāṇaṁ vā pāpakaṁ vā. Viramath’āyasmanto mama vacanāyā” ti. So bhikkhu bhikkhūhi evam-assa vacanīyo, “Mā āyasmā attānaṁ avacanīyaṁ akāsi. Vacanīyam-eva āyasmā attānaṁ karotu. āyasmā'pi bhikkhū vadetu saha-dhammena, bhikkhū'pi āyasmantaṁ vakkhanti saha-dhammena. Evaṁ saṁvaḍḍhā hi tassa bhagavato parisā, yad’idaṁ aññam-añña-vacanena aññam-añña-vuṭṭhāpanenā” ti. Evañ-ca so bhikkhu bhikkhūhi vuccamāno tath’eva paggaṇheyya, so bhikkhu bhikkhūhi yāva-tatiyaṁ samanubhāsitabbo tassa paṭinissaggāya.Yāva-tatiyañ-ce samanubhāsiyamāno taṁ paṭinissajjeyya, icc’etaṁ kusalaṁ. No ce paṭinissajjeyya, saṅghādiseso.

\pdfbookmark[2]{Saṅghādisesa 13}{sd13}
\subsection*{\hyperref[comm13]{Saṅghādisesa 13: Kuladūsakasikkhāpadaṁ}}
\label{sd13}
Bhikkhu pan’eva aññataraṁ gāmaṁ vā nigamaṁ vā upanissāya viharati kula-dūsako pāpa-samācāro. Tassa kho pāpakā samācārā dissanti c’eva suyyanti ca, kulāni ca tena duṭṭhāni dissanti c’eva suyyanti ca. So bhikkhu bhikkhūhi evam-assa vacanīyo, “āyasmā kho kula-dūsako pāpa-samācāro. āyasmato kho pāpakā samācārā dissanti c’eva suyyanti ca, kulāni c’āyasmatā duṭṭhāni dissanti c’eva suyyanti ca. Pakkamat’āyasmā imamhā āvāsā, alan-te idha vāsenā” ti. Evañ-ca so bhikkhu bhikkhūhi vuccamāno te bhikkhū evaṁ vadeyya, “Chanda-gāmino ca bhikkhū, dosa-gāmino ca bhikkhū, moha-gāmino ca bhikkhū, bhaya-gāmino ca bhikkhū, tādisikāya āpattiyā ekaccaṁ pabbājenti, ekaccaṁ na pabbājentī” ti. So bhikkhu bhikkhūhi evam-assa vacanīyo, “Mā āyasmā evaṁ avaca. Na ca bhikkhū chanda-gāmino, na ca bhikkhū dosa-gāmino, na ca bhikkhū moha-gāmino, na ca bhikkhū bhaya-gāmino. āyasmā kho kula-dūsako pāpa-samācāro. āyasmato kho pāpakā samācārā dissanti c’eva suyyanti ca, kulāni c’āyasmatā duṭṭhāni dissanti c’eva suyyanti ca. Pakkamat’āyasmā imamhā āvāsā, alan-te idha vāsenā” ti. Evañ-ca so bhikkhu bhikkhūhi vuccamāno tath’eva paggaṇheyya, so bhikkhu bhikkhūhi yāva-tatiyaṁ samanubhāsitabbo tassa paṭinissaggāya. Yāva-tatiyañ-ce samanubhāsiyamāno taṁ paṭinissajjeyya, icc’etaṁ kusalaṁ. No ce paṭinissajjeyya, saṅghādiseso.

\medskip

\begin{center}
Uddiṭṭhā kho āyasmanto terasa saṅghādisesā dhammā, nava paṭham’āpattikā cattāro yāva-tatiyakā. Yesaṁ bhikkhu aññataraṁ vā aññataraṁ vā āpajjitvā yāvatihaṁ jānaṁ paṭicchādeti, tāvatihaṁ tena bhikkhunā akāmā parivatthabbaṁ. Parivuttha-parivāsena bhikkhunā uttariṁ chā-rattaṁ, bhikkhu-mānattāya paṭipajjitabbaṁ. Ciṇṇa-mānatto bhikkhu, yattha siyā vīsati-gaṇo bhikkhu-saṅgho, tattha so bhikkhu abbhetabbo. Ekena'pi ce ūno vīsati-gaṇo bhikkhu-saṅgho taṁ bhikkhuṁ abbheyya, so ca bhikkhu anabbhito, te ca bhikkhū gārayhā. Ayaṁ tattha sāmīci.

\smallskip

Tatth'āyasmante pucchāmi: Kacci'ttha parisuddhā?\\
Dutiyam'pi pucchāmi: Kacci'ttha parisuddhā?\\
Tatiyam'pi pucchāmi: Kacci'ttha parisuddhā?

\smallskip

Parisuddh'etth'āyasmanto, tasmā tuṇhī, evam'etaṁ dhārayāmī.
\end{center}

\begin{outro}
  Saṅghādises’uddeso niṭṭhito
\end{outro}

\clearpage
