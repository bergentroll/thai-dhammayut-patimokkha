\section{Abbreviations}

% \thispagestyle{empty}

% {\subsectionFmt{Abbreviations used in the text}}
% \bigskip

% {\raggedright
% \fontsize{10}{14}\selectfont

\begin{center}
  \begin{tabular}{@{}lll@{}}
    % [...  \hspace{-0.5mm}\anglebracketright\ & = & Only recited by the leader \\
    \breathmark\ & = & Take a breath \\
  \end{tabular}

  % Wisdom Publication sources: Nikāya and sutta # (eg. DN 1)
  % P.T.S. sources: Nikāya, volume #, page # (eg. D i 1)

  \begin{tabular}{@{}ll@{}}
    Vin   & Vinaya Piṭaka           \\
    DN    & Dīgha Nikāya            \\
    MN    & Majjhima Nikāya         \\
    SN    & Saṁyutta Nikāya         \\
    AN    & Aṅguttara Nikāya        \\
    Khp   & Khuddakapāṭha           \\
    Dhp   & Dhammapada              \\
    Ud    & Udāna                   \\
    Snp   & Sutta Nipāta            \\
    Thag  & Theragāthā              \\
    Ja    & Jātaka                  \\
    Ps    & Paṭisambhidāmagga       \\
    Vibh  & Abhidhamma Vibhaṅga     \\
    A     & Aṭṭhakathā              \\
    Dhs   & Dhammasaṅganī           \\
    A     & Aṭṭhakathā              \\
    A     & Aṭṭhakathā              \\
    MJG   & Mahā-jaya-maṅgala-gāthā \\
    Thai  & Composed...             \\
    Sri L & Composed...             \\
    Trad  & Tradtional...           \\
  \end{tabular}
\end{center}

% \bigskip

% Wisdom Publication sources: Nikāya and sutta # (eg. DN 1)
% P.T.S. sources: Nikāya, volume #, page # (eg. D i 1)

% References to shorter texts consisting of verses such as the Dhammapada, Udāna,
% Itivuttaka, Theragāthā, Therīgāthā or Sutta Nipāta are to the verse number or
% chapter and verse number. The other longer texts are referred to by volume and
% page number of the PTS edition.

% }
